\section{Muestreo aleatorio y teorema del límite central}

\paragraph{Ejemplo}
Supongamos que tratamos de encontrar la edad promedio en una ciudad, digamos Oaxaca. Una manera de hacerlo sería por \emph{fuerza bruta}, es decir, recolectando esta información persona por persona. Pero este método sería muy costoso en términos de infraestructura y tiempo.


En estadística, este es un problemalema común, cuya solución está en el \emph{muestreo aleatorio}:  Tomemos un grupo de 1000 individuos (o 10,000 dependiendo de tu capacidad, obviamente entre más, es mejor) y calculemos la edad promedio en este grupo, a la que denotaremos por $A_{1}.$ 


Repitamos este procedimiento, digamos 100 veces, y denotaremos por $A_{1}, A_{2},...,A_{100}$ el promedio de edades obtenido en cada respectivo intento.

De acuerdo a la \emph{ley de los grandes números}, la cantidad
\begin{align}
	\bar{A}_{100}=\dfrac{A_{1}+...+A_{100}}{100}
\end{align}
es una aproximación muy cercana al promedio real de la edad de los pobladores de la ciudad.


De acuerdo al \emph{teorema del límite central}, si el número de tales muestras es suficientemente grande,
$A_{1},A_{2},...,A_{100}$ estarán distribuidos de manera normal.


\begin{observacion}
	No estamos más interesados en obtener el valor exacto de la edad promedio, si no establecer un \emph{estimador} para la misma. 
	
	En tal caso,
	tenemos que conformarnos con la definición de un \emph{rango de valores} en el que el valor real podría estar.
\end{observacion}


% 
% Dado que hemos supuesto una \emph{distribución normal} para los valores de edad media de estos
% grupos, podemos \emph{aplicar todas las propiedades} de una distribución normal para
% posibilidades de que la edad \emph{promedio} este en algún \emph{intervalo}.
% 
