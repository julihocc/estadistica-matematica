\section{Teorema de Bayes}

Si tanto $ P(A) $ como $ P(B) $ son diferentes de cero, entonces, a partir de la definición de probabilidad condicional, podemos deducir que 
\begin{align}
	P(A \cap B) &=P(A)\cdot P(B|A)\\
	P(B \cap A) &=P(B)\cdot P(A|B),
\end{align}
y como $ P(A\cap B)= P(B\cap A) $, concluimos 
\begin{align}
	P(A)\cdot P(B|A) = P(B)\cdot P(A|B)
\end{align}
y por tanto 
\begin{align}
	P(A|B)=\dfrac{P(A)\cdot P(B|A)}{P(B)}.
\end{align}
El resultado es conocido como \emph{teorema de Bayes.}

\subsection{Generalizaciones}

El resultado anterior se puede extender de la siguiente manera: Como 
\begin{align}
	\begin{cases}
		(B\cap A) \cup (B \cap A') &= B\\
		(B\cap A) \cap (B \cap A') &= \emptyset\\
	\end{cases},
\end{align}
entonces
\begin{align}
	P(B) & = P\left( (B\cap A) \cup (B \cap A') \right)\\
	&= P(B\cap A) + P(B\cap A')\\
	&= P(B|A)P(A) + P(B|A')P(A'),
\end{align}
de manera que 
\begin{align}
	P(A|B)=\dfrac{P(A)\cdot P(B|A)}{P(B|A)P(A) + P(B|A')P(A')}.
\end{align}

Podemos generalizar este concepto usando una partición $ \seq{E_i}{i=0}{N} $ del espacio muestral $ S \supset A, B$. En ese caso
\begin{align}
	A = \bigcup_{i=0}^{N}(A\cap E_i)
\end{align}
y por tanto
\begin{align}
	P(B) &= \sum_{i=0}^{N} P(B\cap E_i)\\
	&= \sum_{i=0}^{N} P(B|E_i)P(E_i).
\end{align}
A este resultado lo llamaremos \emph{regla de la cadena.}

Ahora bien, de manera similar al primer caso del teorema de Bayes, concluímos que 
\begin{align}
	P(B|E_j)P(E_j)= P(E_j|B)P(B),
\end{align}
de forma que utilizando la regla de la cadena, obtenemos
\begin{align}
	P(E_j|B)&= \dfrac{P(B|E_j)P(E_j)}{P(B)}\\
	&=\dfrac{P(B|E_j)P(E_j)}{\sum_{i=0}^{N} P(B|E_i)P(E_i)}
\end{align}

\begin{ejemplo}
	Una planta productora de gelatinas cuenta con tres máquinas empacadoras. Así la distribución de volumen de empaque se realiza de la siguiente manera:
	\begin{itemize}
		\item Máquina 1: 38\%
		\item Máquina 2: 32\%
		\item Máquina 3: 30\%
	\end{itemize}
	
	De esta manera, la probabilidad de que el empaque salga defectuoso es de 11\%, 15\% y 14\%, respectivamente por cada máquina. 
	
	La gerencia de producción de la planta está interesada en conocer cuál es la probabilidad de que si se selecciona una unidad al azar y es defectuoso, esta se haya empacado en la máquina 2. 

	Denotemos por $ M_i $ el evento de que una unidad de gelatina se haya empacado en la $ i $-ésima máquina, mientras que $ D $ es el evento de que la unidad sea defectuosa. 
	
	De acuerdo al problema 
	\begin{align}
		P(M_1)&=0.38\\
		P(M_2)&=0.32\\
		P(M_3)&=0.30\\
		P(D|M_1)&=0.11\\
		P(D|M_2)&=0.15\\
		P(D|M_3)&=0.14
	\end{align}
	
	Entonces 
	\begin{align}
		P(M_2|D) &= \dfrac{P(D|M_2)P(M_2)}{	P(D|M_1)P(M_1)+P(D|M_2)P(M_2)+P(D|M_3)P(M_3)
		}\\
		&= \dfrac{(0.15)(0.32)}{
			(0.11)(0.38)+(0.15)(0.32)+(0.14)(0.30)	
		}\\
		&=0.3642=36.42\%
	\end{align}
\end{ejemplo}