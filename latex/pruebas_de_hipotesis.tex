\section{Pruebas de hipótesis}
% 
% El concepto que acabamos de comentar en la sección anterior se utiliza para una
% técnica en estadística, llamada \emph{prueba de hipótesis}.
% 


En la prueba de hipótesis, asumimos una
premisa inicial (generalmente relacionada con el valor del estimador) denominada \emph{hipótesis nula}  y
trataremos de ver si es cierta o no aplicando.



Tenemos otra premisa llamada \emph{hipótesis alternativa}, la cuál es la negación de la hipótesis nula.


\paragraph{Hipótesis nula vs. alternativa}
%  Hay una forma para decidir cuál será la hipótesis nula y cuál será la hipótesis alternativa. 
%
% La hipótesis nula es la premisa inicial o algo que
% podemos suponer que es cierto. 
%
% Por el contrario, la hipótesis alternativa es algo de lo que no estamos seguros que podría ser cierto.
% 
%
% 
Cuando alguien está haciendo una \emph{investigación} cuantitativa para calibrar el valor de un estimador,  el \emph{valor conocido} del parámetro se toma como \emph{hipótesis nula},  mientras que el \emph{nuevo valor} encontrado (de la investigación) se toma como la \emph{hipótesis alternativa}.



En nuestro caso (encontrar la edad media de nuestra ciudad), un investigador puede afirmar que la edad
\emph{menor que 35}. Esto puede servir como la \emph{hipótesis nula}.


Si una nueva agencia afirma
que es \emph{mayor que 35}, entonces se puede denominar como la \emph{hipótesis alternativa}.

