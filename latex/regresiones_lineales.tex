\section{Introducción}

En esta unidad, trataremos con una técnica básica de modelación predictiva llamada \emph{regresión lineal}, la cuál permite crear un modelo a partir de una base de datos histórica.


Nuestro propósito es entender las matemáticas detrás de la regresión lineal e ilustrar sus resultado a través de su implementación en varias bases de datos.

\paragraph{Mapa de ruta}
\begin{itemize}
	\item Las matemáticas detrás de la regresión lineal.
	\item Implementación de la regresión lineal con Python.
	\item Interpretación de los parámetros resultantes.
	\item Validación del modelo.
	\item Manejo de Problemas relacionados con regresión lineal.
\end{itemize}


\paragraph{Modelos matemáticos}
Un \emph{modelo matemático/estadístico/predictivo} es una ecuación matemática que consiste en \emph{entradas} que producen \emph{salidas} cuando el valor de las variables entrantes se introduce en el modelo.

\paragraph{Ejemplo}
Por ejemplo, supongamos que el precio $P$ de una casa es \emph{linealmente dependiente} en su tamaño $S$, comodidades $A$ y disponibilidad de transporte $T$.



La ecuación correspondiente sería
\begin{align}
	P = a_{1}\times S+ a_{2} \times A + a_{3} \times T
\end{align}



Esta ecuación es llamado el \emph{modelo} y los coeficientes $a_{1},a_{2},a_{3}$ son sus parámetros.



La variable $P$ es resultado predicho, mientras que $S,A,T$ con las variables de entrada, que son datos conocidos.


Sin embargo, los parámetros $a_{i}$ deben ser estimados a partir de los datos históricos.


Una vez que estos parámetros son determinados, el modelo está listo para ser problemaado.

