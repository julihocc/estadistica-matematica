\section{Medidas de tendencia central}

\subsection{\'Indice y subíndices}
El símbolo $X_{j}$ representa cualquiera de los  valores $X_{1},X_{2},X_{3},...$ que puede tomar la variable discreta $X.$


El símbolo $j$ denota cualquiera de los números naturales $1,2,3,...$ y se le llama \emph{índice} (o a veces \emph{subíndice} o también \emph{contador}).




\begin{definicion}[Sumatoria]
	\begin{align}
		\sum_{j=1}^{N}X_{j}=X_{1}+...+X_{N}
	\end{align}
\end{definicion}



\begin{ejemplo}
	\begin{itemize}
		\item $\displaystyle \sum_{k=1}^{N}X_{k}Y_{k}=
		X_{1}Y_{1}+...+X_{N}Y_{N}$
		\item $\displaystyle \sum_{i=1}^{N} aX_{i}=
		aX_{1}+...+aX_{N}=a\sum_{n=1}^{N}X_{n}.$
		\item 
		Si $a,b$ son constantes, demuestre que
		\begin{align}
			\sum \left( aX+bY \right)=a\sum X + b\sum Y.
		\end{align}
	\end{itemize}
	
\end{ejemplo}



\begin{observacion}
	Cuando se \emph{sobrentiende} que el contador $j$ \emph{corre} sobre los números $1,2,...,N,$ escribimos $\sum X_{j}$ o simplemente $\sum X$ en lugar de $\sum_{j=1}^{N}.$
\end{observacion}


\subsection{Promedio}
Un \emph{promedio} es un valor representativo de un conjunto de datos que tiende a encontrarse en el centro de dicho conjunto. Por esta razón, también se le conoce como \emph{medidas de tendencia central.}



Se pueden definir varios tipo de promedios: 
\begin{itemize}
	\item Media aritmética;
	\item mediana;
	\item moda;
	\item media geométrica;
	\item media armónica.
\end{itemize}



\begin{observacion}
	Cada medida de tendencia central tiene ventajas y desventajas de acuerdo al tipo de datos y el propósito del uso.
\end{observacion}


\subsection{Media aritmética}

\begin{definicion}[Media aritmética]
	\begin{align}
		\label{3.1}
		\bar{X} =\dfrac{X_{1}+...+X_{N}}{N} = \dfrac{\sum_{j=1}^{N}X_{j}}{N}=\dfrac{\sum X}{N}
	\end{align}
\end{definicion}



\begin{ejemplo}
	Calcula la media de $8,3,5,12,10$.
\end{ejemplo}

\lstinputlisting[language=python]{../code/medida-tendencia-central/ejemplo-2-2/ejemplo_2_2.py}


Si los números $X_{1},X_{2},...,X_{k}$ se presentan con \emph{frecuencias} $f_{1}, f_{2},...,f_{k}$ respectivamente su media aritmética es
\begin{align}
	\label{3.2}
	\bar{X}=\dfrac{f_{1}X_{1}+...+f_{k}X_{k}}{f_{1}+...+f_{k}}=\dfrac{\sum fX}{\sum f}=\dfrac{\sum fX}{N}.
\end{align}
dónde $N=\sum f$ es la \emph{suma de frecuencias} o \emph{total de casos.}


\begin{ejemplo}
	Si $5,8,6,2$ se presentan con frecuencias $3,2,4,1$ respectivamente, su media aritmética es...
\end{ejemplo}
\lstinputlisting[language=python]{../code/medida-tendencia-central/ejemplo-2-3/ejemplo_2_3.py}
 
Algunas veces, a los números $X_{1},...,X_{k}$ se les asignan ciertos \emph{factores de ponderación} o \emph{pesos} $w_{1},...,w_{k},$ tales que \begin{align}
	\begin{cases}
		0\% \leq w_{i}\leq 100\% \\
		\sum w_{i} = 100\%
	\end{cases}
\end{align}


\begin{definicion}[Media ponderada]
	Si $w_{1},..,w_{k}$ son \emph{pesos} tales que $0\leq w_{i}\leq 1$ y $\sum w_{i}=1,$ entonces la correspondiente media (aritmética) ponderada de los números $X_{1},...,X_{k}$ es
	\begin{align}
		\bar{X}= \dfrac{w_{1}X_{1}+...+w_{k}X_{k}}{w_{1}+...+w_{k}}=\dfrac{\sum wX}{\sum w}=\sum wX.
	\end{align}
\end{definicion}



\begin{ejemplo}
	Si en una clase, al examen final se le da el triple del valor que a los exámenes parciales y un estudiante obtiene 85 en el final y 70 y 90 en los dos exámenes parciales, obtener su media ponderada.
\end{ejemplo}



\begin{enumerate}
	\item Si $w_{i}=\frac{1}{N},$ obtenemos la media aritmética usual. 
	\item Si $w_{i}=\frac{f_{i}}{N},$ obtenemos la fórmula \eqref{3.2}.
\end{enumerate}

Cuando los números son muy grandes, se suele utilizar un pivote $P:$ 
$$
\bar{X}=P+\dfrac{\sum f_{i}d_{i}}{N},
$$
donde $d_{i}=X_{i}-P.$

En ocasiones, utilizaremos la notación
\begin{align}
	\bar{d}=\dfrac{\sum f_{i}d_{i}}{N},
\end{align}
de manera que $\bar{d}$ es la \emph{desviación promedio} y $\bar{X}=P+\bar{d}.$



\begin{observacion}
	
	Para datos agrupados, $X_{i}$ se escoge como la marca de la $i-$ésima clase.
\end{observacion}


\subsection{La mediana}
La mediana $\til{X}$ de un conjunto de números acomodados en un orden de magnitud (es decir, en una ordenación) es el valor central o la media de dos valores centrales.


\begin{ejemplo}
	\begin{itemize}
		\item La mediana de la lista de números $5, 4, 3, 8, 6, 2, 5, 2$ es... 
		\item La mediana de la lista de números $3, 9, 1, 1, 4, 1, 3, 2, 4$ es..
	\end{itemize}	
\end{ejemplo}

\lstinputlisting{../code/medida-tendencia-central/ejemplo-2-5/ejemplo_2_5.py}



\begin{definicion}[Mediana para datos agrupados]
	\begin{align}
		\texttt{Mediana}=L+\left(
		\dfrac{ \dfrac{N}{2}-\sum_{{C<C_{M}}}f}{f_{C_{M}}}
		\right)
	\end{align}
	donde 
	\begin{itemize}
		\item $L$ es la frontera inferior de la clase mediana, es decir, de la clase que contiene la mediana;
		\item $N$ es la frecuencia total; 
		\item $\sum_{{C<C_{M}}}f$ suma de las frecuencias de todas las clases anteriores a la clase mediana; 
		\item $f_{C_{M}}$ es la frecuencia de la clase mediana.
	\end{itemize}
	
\end{definicion}


\subsection{Moda}
La moda de una lista de números es un valor que se presenta con la mayor frecuencia $f>1$.  \emph{La moda no es necesariamente existe ni es única.}


\begin{ejemplo}
	\begin{itemize}
		\item La moda de la lista $2,2,5,7,9,9,9,10,10,11,12,18$ es...  En este caso, diremos que la lista es \emph{unimodal.}
		\item ?`Cuál es la moda de la lista $3,5,8,0,12,15,16$? 
		\item ?`Cuál es la moda de la lista $3,8,8,8,15,15,15$?  En este caso diremos que la lista es \emph{bimodal.}
	\end{itemize}
	
\end{ejemplo}



\begin{definicion}[Moda para datos agrupados]
	\begin{align}
		\texttt{Moda}=
		L + \left( \dfrac{\Del_{1}}{\Del_{1}+\Del_{2}} \right)c
	\end{align}
	donde 
	\begin{enumerate}
		\item $L:$ Frontera inferior de la clase modal, es decir, de la clase que contiene la moda.
		\item $\Del_{1}:$ Exceso de frecuencia modal sobre la frecuencia en la clase inferior inmediata. 
		\item $\Del_{2}:$ Exceso de frecuencia modal sobre la frecuencia en la clase superior inmediata. 
		\item $c:$ Amplitud del intervalo de la clase modal.
	\end{enumerate}
	
\end{definicion}


%%%%%%%%%%%%%%%%%%5
%%%%%%%%%%%%%%%%%%%%%5

\subsection{Paquetes especializados}

\texttt{Numpy} es el módulo numérico de Python. Nos permite hacer cálculos numéricos con gran velocidad. Con \texttt{Numpy}, es sencillo calcular  la media aritmética sobre los elementos de un arreglo.

\begin{lstlisting}[language=Python]
	a = np.array([[1, 2], [3, 4]])
	print np.mean(a)
	#2.5
	print np.mean(a, axis=0)
	#array([ 2.,  3.])
	print np.mean(a, axis=1)
	#array([ 1.5,  3.5])
\end{lstlisting}


También podemos calcular la mediana.
\begin{lstlisting}[language=Python]
	import numpy as np
	
	a = np.array([[10, 7, 4], [3, 2, 1]])
	print a
	#array([[10,  7,  4],6[ 3,  2,  1]])
	print np.median(a)
	#3.5
	print np.median(a, axis=0)
	#array([ 6.5,  4.5,  2.5])
	print np.median(a, axis=1)
	#array([ 7.,  2.])
	
	m = np.median(a, axis=0)
	out = np.zeros_like(m)
	print np.median(a, axis=0, out=m)
	#array([ 6.5,  4.5,  2.5])
	print m
	#array([ 6.5,  4.5,  2.5])
	b = a.copy()
	print np.median(b, axis=1, overwrite_input=True)
	#array([ 7.,  2.])
	
	assert not np.all(a==b)
	b = a.copy()
	print np.median(b, axis=None, overwrite_input=True)
	#3.5
	assert not np.all(a==b)
\end{lstlisting}

Sin embargo, no hay una forma sencilla de calcular la moda con \texttt{Numpy}. Por lo que usaremos otro módulo llamado \texttt{SciPy} es una biblioteca open source de herramientas y algoritmos matemáticos para Python.

SciPy contiene módulos para optimización, álgebra lineal, integración, interpolación, funciones especiales, FFT, procesamiento de señales y de imagen, resolución de ODEs y otras tareas para la ciencia e ingeniería. Está dirigida al mismo tipo de usuarios que los de aplicaciones como MATLAB, GNU Octave, y Scilab.\sidenote{\href{https://es.wikipedia.org/wiki/SciPy}{https://es.wikipedia.org/wiki/SciPy}}

\begin{lstlisting}[language=Python]
	import numpy as np
	from scipy import stats
	
	a = np.array([3,5,6,5,6,5,6,6,3,1,5])
	print stats.mode(a)
	# ModeResult(mode=array([5]), count=array([4]))
\end{lstlisting}


