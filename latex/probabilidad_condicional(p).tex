\section*{Problemas}

\begin{problema}
	\label{exmp:1.13}
	Encontrar la probabilidad de que una solo lanzamiento de un dado resulte en un número menor que $4$ si
	\begin{enumerate}
		\item no hay más información; 
		\item se sabe que el lanzamiento resultó en un número impar.
	\end{enumerate}
	
\end{problema}

\begin{problema}
	\label{solved:1.16}
	Demuestre el teorema de Bayes.
\end{problema}


\begin{problema}
	\label{solved:1.15}
	La caja $I$ contiene 3 canicas rojas y 2 azules, mientras que la caja $II$ contiene $8$ canicas rojas y 8 azules. Una moneda se lanza: Si cae un sol, se escoge una moneda de la caja $I$ y si cae reverso, de la caja $II.$ Encuentre la probabilidad de obtener una canica roja.
\end{problema}


{}
\begin{problema}
	\label{solved:17}
	Supongamos que en el problema anterior, quien lanza la moneda no revela si ha caído reverso o sol (de manera que la caja de la que se obtiene la canica no se revela) pero revela que una canica roja se ha obtenido.?`Cual es la probabilidad de haber obtenido un sol?
\end{problema}